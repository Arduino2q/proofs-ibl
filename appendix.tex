\appendix
\chapter{Peano axioms}
\newcommand{\axiom}[1]{\textsc{Axiom~#1}\hspace{.5ex}}

Particularly in the 
first chapter a person struggles with when to consider a 
statement sufficiently justified 
and soon comes to wonder what the axioms are like.
Here we give the most often used axiom system for the natural numbers, to
convey a sense of that. 

This system was 
introduced by Dedekind in 1888 and tuned by Peano in 1889.
In addition to the usual logical and set symbols such as $=$ and $\in$,
with the traditional properties, 
our language will use at least two symbols, $0$ and $S$, whose 
properties are limited by the conditions below.

\begin{ax}[Existence of a natural number]
The constant~$0$ is a natural number.
\end{ax}

\begin{ax}[Arithmetical properties]
The \definend{successor} function~$S$ has these properties.
\begin{exes}
\item \notetext{Closure} For all $a\in\N$, its successor~$S(a)$ is also 
  a natural number.
\item \notetext{One-to-one} For all $a,b\in\N$, if $S(a)=S(b)$ then $a=b$.
\item \notetext{Almost onto}
  For all $a\in\N$, if $a\neq 0$ then there is a $b\in\N$ with $S(b)=a$.
  In contrast, no $c\in\N$ has $0$ as a successor.  
\end{exes}
\end{ax}

These properties give infinitely many natural numbers:
$0$, $S(0)$, $S(S(0))$, etc.
Of course, the notation $0$, $1$, $2$, etc., is less clunky.

\begin{ax}[Induction]
  Suppose that $K$ is a set satisfying both (i)~$0\in K$
  and (ii)~for all $n\in\N$, if $n\in K$ then $S(n)\in K$.
  Then $K=\N$.
\end{ax}

\noindent (In this book we use an induction variant that changes
condition~(ii) to: for all $k\in\N$,  
if $n\in K$ for $0\leq n\leq k$ then $S(k)\in K$.
Condition~(ii) above is often called \definend{weak induction} while
the version we use is \definend{strong induction}.
There are technical differences but for our purposes
the two variants are interchangeable. 
We prefer the strong variant because while it is 
more awkward to state, it is sometimes easier to apply.)
% http://mathoverflow.net/questions/37944/induction-vs-strong-induction

From those axioms we can for instance 
define addition by recursion using successor
\begin{equation*}
  \add(a,n)=
  \begin{cases}
    a             &\text{if $n=0$}  \\
    S(\add(a,m))  &\text{if $n=S(m)$} 
  \end{cases}
\end{equation*}
and then define multiplication by recursion using addition.
\begin{equation*}
  \mul(a,n)=
  \begin{cases}
    0             &\text{if $n=0$}  \\
    \add(\mul(a,m),a)  &\text{if $n=S(m)$} 
  \end{cases}
\end{equation*}
