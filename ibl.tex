\documentclass{ibl}
\newcommand{\grf}[2]{\begin{tabular}{@{}c@{}}\includegraphics{#1} \\ #2\end{tabular}}

\frontmatter
\begin{document}    
\vspace*{\fill}
\begin{center}
  \setlength{\unitlength}{1in}
  \newcommand{\enormoussize}{\fontsize{30pt}{20pt}\selectfont}
  {\enormoussize\scshape \makebox[0em][l]{\raisebox{24pt}{\large\itshape \hspace*{15pt}An Inquiry-Based}}{\color{darkii}Introduction to Proof}}
\end{center}
\vspace*{\fill}
\begin{flushright}
  \large\color{darki}
  \begin{tabular}{@{}l@{}}
  Jim Hef{}feron  \\
  version 0.5
  \end{tabular}
\end{flushright}

\newpage
\vspace*{\fill}
\begin{center}
  \renewcommand{\arraystretch}{1.07}
  {\large\textsc{Notation}} \\[3ex]
  \begin{tabular}{r|l}
    $\Z$    &integers; $\set{\ldots, -2,-1,0,1,2,\ldots}$  \\
    $\N$    &natural numbers; $\set{0,1,2,\ldots}$  \\
    $\Z^+$  &positive integers; $\set{1,2,\ldots}$  \\
    $\Q$    &rational numbers  \\
    $\R$    &real numbers     \\
    $a\divides b$  &$a$ divides~$b$ \\
    $a\bmod b$  &the remainder when $a$ is divided by $b$ \\
    $a\equiv c\pmod b$ &$a$ and $c$ have the same remainder when divided by $b$ \\
    $\gcd(a,b)$ &greatest common divisor of $a$ and~$b$  \\
    $\lcm(a,b)$ &least common multiple of $a$ and~$b$  \\
    $a\in A$  &$a$ is an element of $A$  \\
    $\emptyset$  &the empty set  \\
    $A\subseteq B$ &$A$ is a subset of~$B$  \\
    $\charfcn{A}$  &characteristic function of the set~$A$ \\
    $\setcomp{A}$  &complement of the set~$A$  \\
    $A\union B$, $A\intersection B$ &union and intersection of the two sets \\
    $A- B$, $A\symdiff B$ &difference and symmetric difference of the sets \\
    $|A|$  &order of the finite set $A$; number of elements \\ 
    $\powerset(A)$  &power set of $A$; set of all of $A$'s subsets \\
    $\seq{x_0,x_1,\ldots}$,$\lh(\seq{x_0,x_1,\ldots})$ &sequence, length of the sequence   \\
    $A_0\times A_1\times \cdots \times A_{n-1}$  &Cartesian product \\
    $A^n$  &Cartesian product of the set with itself  \\
    $\map{f}{D}{C}$  &function with domain~$D$ and codomain~$C$ \\
    $\map{\identity}{D}{D}$  &identity map; $\identity(d)=d$ \\
    $\restrictionmap{f}{B}$  &restriction of $f$ to a subset of the domain \\
    $f^{-1}(c)$, $f^{-1}(A)$  &inverse image of an element or subset of the codomain \\
    $\composed{g}{f}$  &composition of the functions  \\
    $f^{-1}$  &function inverse to~$f$  \\
    $x\equiv y\pmod R$  &$(x,y)\in R$ where $R$ is an equivalence \\ 
    $\eqclass{x}$  &equivalence class containing~$x$  \\
    $A\sim B$  &the two sets have the same cardinality
  \end{tabular}
\end{center}
\vspace*{\fill}





\chapter*{Preface}

This is a course in Mathematical proof. 
Its audience is majors who are US sophomores, although since
it requires only high school mathematics
it can be used with first year students.

This course is Inquiry-Based (sometimes called Moore method).
That is, this text is a sequence of exercises
along with definitions and a few 
remarks.
The class works through those mostly by
proving the exercises, or sometimes by providing examples or counterexamples.
Thus, the students directly grapple with the material, 
with the instructor only lightly guiding, 
and together work though misunderstandings, 
sometimes stumbling in the dark and occasionally
having beautiful flashes of insight.
With students at this level, 
this approach is the best way to build mathematical maturity.

The material starts with elementary number theory, 
then moves to sets, functions, and relations. 
There is an optional brief final chapter on cardinality.
We approach these naively, meaning without a systematic development from
axioms.


\medskip
\noindent\textsc{Background topics.}
We start with elementary number theory, not truth tables, 
for the same reason
that Little League starts by playing ball, not with the rulebook.
Math majors readily take to proving things about
divisibility and primes, enjoying the 
intellectual and verbal challenge, 
whereas a month of preliminary material does not 
spark enthusiasm.
This also explains why we use some terms, such as an element of a set, before
their definition.

But the underlying logic is good stuff and besides, they need it.
I have daily classes begin with some students going to the 
board to write their answers to the homework and 
I use these ten minutes for a slide or two that covers issues from
the class's current work, such as the truth table 
definition of implication.
This just-in-time approach keeps 
the class's attention on proving things, but they get the
background that they need at the moment when they can appreciate it.

The only exception is that this book covers induction as a separate
topic, since it is often new for students and since it takes 
the class's full attention.



\medskip
\noindent\textsc{License.}
This material is Free; see the details  
at \url{http://joshua.smcvt.edu/proofs}.
This includes the book's \LaTeX{} source, so instructors can tailor the
material to their students.  
Also on that page are the slides that I use
and, for instructors new to this approach, 
a brief description of what classroom technique works for me.


\vspace{\fill}
\noindent\parbox{.95\textwidth}{\raggedright\textit{At the first meeting of the class Moore would define the basic terms and either challenge the class to discover the relations among them, or, depending on the subject, the level, and the students, explicitly state a theorem, or two, or three. Class dismissed. Next meeting: "Mr Smith, please prove Theorem 1. Oh, you can't? Very well, Mr Jones, you? No? Mr Robinson? No? Well, let's skip Theorem 1 and come back to it later. How about Theorem 2, Mr Smith?" Someone almost always could do something. If not, class dismissed. It didn't take the class long to discover that Moore really meant it, and presently the students would be proving theorems and watching the proofs of others with the eyes of eagles.}\hspace{2em} --Paul Halmos}

\vspace{.2in}
\noindent\parbox{.95\textwidth}{\textit{It's a kind of art that may change lives.}\hspace{2em} --Peter Schjeldahl}
 
\vspace*{\fill}
\begin{flushright}
  \begin{tabular}{@{}l@{}}
  Jim Hef{}feron  \\
  Saint Michael's College  \\
  2013-Spring
  \end{tabular}
\end{flushright}




\mainmatter
\pagestyle{bodypage}
\chapter{Number Theory}

We begin with elementary results about the
\definend{integers\/} $\Z=\set{..,-2,-1,0,1,2,..}$.
In this chapter, ``number'' means integer.
In some cases, results refer to the 
\definend{natural numbers\/} $\N=\set{0,1,2,..}$.
% (Be aware that some mathematicians mistakenly 
% perceive that the natural numbers do not include~$0$.)



% .....................................
\section{Divisibility}

\begin{df}
  For two integers $d,n$ we say that
  $d$ \definend{divides}~$n$
  if there is an integer~$k$ such that $dk=n$.
  Here, $d$~is the \definend{divisor}, 
  $n$~is the \definend{dividend},
  and~$k$ is the \definend{quotient}.
  (Alternative wordings are:
  $d$~\definend{is a factor of}~$n$,
  or $d$~\definend{goes evenly into}~$n$,
  or $n$~\definend{is a multiple of}~$d$.)
  We denote the relationship as
  $d\divides n$ if $d$~is a divisor of~$n$
  or as $d\ndivides n$ if it is not.
\end{df}

\begin{df}
  A natural number is \definend{even} if it is a multiple of~$2$.
  Otherwise it is \definend{odd}.
  (Alternative wording is that the number \definend{has even parity}
  or \definend{has odd parity}.)
\end{df}

% \begin{nt}
  The notation $d\divides n$ signifies a relationship between two numbers.
  It is different than the fraction $d/n$, which is a number.
  We can sensibly ask ``Does $2$ divide~$5$?''\spacefactor=1000\
  but the question ``Does $2/5$?''\spacefactor=1000\ is not sensible.   
% \end{nt}

\begin{ex}[Interaction of parity with sign]
\pord.
\begin{exes}
\item If a number is even then its negative is even.
\item If a number is odd then its negative is odd.
\item If $d\divides a$ then $-d\divides a$ and $d\divides -a$.
\item If $d\divides a$ then $d\divides\absval{a}$.
\item For any $a\neq 0$, its largest divisor is $\absval{a}$.
\end{exes}  
\end{ex}

\begin{ex}[Interaction of parity with addition]\pord.
\begin{exes}
\item The sum of two evens is even.
\item The difference of two evens is even.
\item The sum of two odds is odd.
\item The difference of two odds is odd.
\item Generalize to arbitrary divisors~$d$.
\end{exes}  
\end{ex}

\begin{ex}
Find a counterexample to:
where $a,b\in\N$, then $a+b$ is even if and only if $a-b$ is even.
Patch the statement (alter it slightly to make it true).
\end{ex}

\begin{ex}[Interaction of parity with multiplication]\pord.
\begin{exes}
\item The product of two evens is even.
\item The quotient of two evens, if it is an integer, is even.
\item The product of two odds is odd.
\item The quotient of two odds, if it is an integer, is odd.
\item Generalize to arbitrary divisors~$d$.
\end{exes}  
\end{ex}

\begin{ex}[Divisibility Properties]
Let $d$, $m$, and $n$ be integers.
\pord.
\begin{exes}
\item \notetext{Reflexivity} Every number divides itself.
\item Every number divides $0$ while
  the only number that $0$ divides is itself.
\item Every number is divisible by $1$ while
  the only numbers that divide $1$ are $1$~itself and $-1$.
\item \notetext{Transitivity} If $d\divides n$ and $n\divides m$ 
   then $d\divides m$.
\item \notetext{Cancellation} 
  For $d,n\in\Z$ we have $d\divides n$ if and only if 
  $ad\divides an$ for any~$a\in\Z$.
\item \notetext{Comparison} 
  For $d,n\in\Z^+$, if $n$ is a multiple of~$d$ then $n\geq d$.
\item If $d\divides a$ and $d\divides b$ then $d\divides (a+b)$.  
\end{exes}
\end{ex}

\begin{ex}
Extend Comparison to all of $\Z$.  
\end{ex}

\begin{ex}
Give the converse of the Reflexivity property: what conclusion can you make
if $a\divides b$ and~$b\divides a$?
\end{ex}

\begin{ex}
Suppose $a,b,c\in\Z$.
\begin{exes}
\item If $a\divides b$ then $a\divides bc$ for all integers~$c$.
\item If $a\divides b$ and $a\divides c$ then $a$ divides the 
  sum~$b+c$ and difference~$b-c$.
\item Generalize.
\end{exes}
\end{ex}




% .....................................
\section{Interlude: induction}
Results in the prior section need only proof techniques that come naturally
to people with a mathematical aptitude.
But some results to follow require a technique 
that is not as natural, mathematical induction.

For that reason, we now pause to introduce induction.
We start with exercises about sums 
(but note that induction is not about sums,
they only make good exercises).
For instance, most people have noticed that the odd numbers sum to 
perfect squares: $1+3=4$, $1+3+5=9$, $1+3+5+7=16$, etc.,
so we look to prove 
the statement, 
``The first $n$ odd numbers $1$, $3$, $5$, $\ldots$, $2n-1$, add to $n^2$.'' 

Our induction proofs have two steps, two separate things to check.
For the base step we show that the statement holds for a first number;
below we will verify the truth of the statement for $n=1$.
The inductive step is more subtle;
we will verify the following implication.
\begin{equation*}
  \begin{tabular}{l} 
  \textit{If} the statement holds from the base case up to and including 
   the $n=k$ case \\
  \textit{then} the statement also holds in the~$n=k+1$ case.
  \end{tabular}
  \tag{$*$}
\end{equation*}
Doing both steps  
proves that the statement is true for all~$n\geq 1$.
For one thing, the base case    
directly verifies the statement for the number~$1$.
Next, because the inductive step verifies the implication~($*$) for all $k$, 
that implication applies when~$k=1$, where it gives 
that the statement is true for the number~$2$. 
Now, with both $n=1$ and~$n=2$ established, 
apply ($*$) again to conclude that the statement is true for the number~$3$.
In this way, we bootstrap to all numbers greater than or equal to $1$.
Below is a sample proof, doing both steps.

\begin{proof}
  We show that $1+3+\cdots+(2n-1)=n^2$ by induction.
  For the $n=1$ base step note that on the left the sum has one term, $1$,
  which equals the value on the right, $1^2$.

  For the inductive step assume that the 
  formula is true for cases $n=1$, $n=2$, \ldots, $n=k$, and 
  consider the $n=k+1$ case.
  The sum is $1+3+\cdots+(2k-1)+(2(k+1)-1)=1+3+\cdots+(2k-1)+(2k+1)$.
  By the inductive hypothesis the statement is true in the $n=k$ case
  so we can substitute.
  \begin{equation*}
    1+3+\cdots+(2k-1)+(2k+1)=k^2+(2k+1)=(k+1)^2
  \end{equation*}
  This is the required expression for the case~$k+1$.
\end{proof}

\begin{ex}
Prove by induction.
\begin{exes}
\item $1+2+\cdots+n=n(n+1)/2$
\item $1+4+9+\cdots+n^2=n(n+1)(2n+1)/6$
\item $1+8+27+\cdots+n^3=n^2(n+1)^2/4$
\end{exes}
\end{ex}

\begin{ex}
Prove by induction that
$1+2+4+8+\cdots+2^{n-1}=2^n-1$.  
\end{ex}

\begin{ex}
Prove by induction.
\begin{exes}
  \item \notetext{Arithmetic series}
       $b+(b+a)+(b+2a)+\cdots+(b+na)=(2b+na)\cdot n/2$
  \item \notetext{Geometric series}
        $1+r+r^2+r^3+\cdots+r^n=(1-r^{n+1})/(1-r)$ for any $r\neq 1$.
\end{exes}
\end{ex}

\begin{ex}
Prove by induction that $n<2^n$ for all $n\in\N$.  
\end{ex}

\begin{ex}
Prove that for all $n\geq 2$, sums can be reversed
$a_1+a_2+\cdots+a_n=a_n+\cdots+a_1$.   
\end{ex}

\begin{ex}
Prove by induction.
\begin{exes}
\item For all $n\geq 2$ the number $6$ divides the number $n^3-n$.
\item If $n\in\N$ then $4\divides 13^n-1$.
\item If $n\in\N$ then
    $(1+\sfrac{1}{1})\cdot(1+\sfrac{1}{2})\,\cdots\,(1+\sfrac{1}{n})=n+1$.
\end{exes}
\end{ex}

Sometimes when you have finished proving something by induction, 
while you now are sure that the statement is true, 
you can be left with the sense
that you are no nearer knowing really \emph{why} it is true.



% .....................................
\section{Division}
We can generalize divisibility.

\begin{ex}
\begin{exes}
\item Where $a$ and $b>0$ are integers, there are integers $q$ and~$r$
such
that $a=bq+r$ and
$r\in\set{0,..,b-1}$.
\item
Prove that the quotient and remainder are unique.
\item 
What to do if $b\leq 0$?
\end{exes}
\end{ex}

\begin{ex}
When is $r=0$? 
\end{ex}

\begin{df}
In the $a=bq+r$ equation above, $a$ is the \definend{dividend} and
$b$ is the \definend{divisor}, while 
$q$ is the \definend{quotient} and~$r$ is the \definend{remainder}.  
\end{df}

From grade school we know an effective procedure, an algorithm, to 
determine the quotient and remainder.

\begin{df}
Where $b>0$ then the \definend{modulus} $a\bmod b$ 
is the remainder when $a$ is divided by~$b$.
Two numbers $a,c$ are \definend{congruent modulo $b$} $a\equiv c\pmod b$ 
if they have the same modulus with respect to~$b$.
\end{df}

\begin{ex}\pord
\begin{exes}
\item $a\bmod b=b\bmod a$
\item $a\bmod b=-a\bmod b$
\item $b\bmod b=1$    
\end{exes}
\end{ex}

\begin{ex}
Prove that $a\equiv c\pmod b$ if and only if $b\divides (a-c)$.  
\end{ex}

\begin{ex}
Let $a,b,c,d,m$ be integers with $m>0$ and
$a\equiv b\pmod m$ and $c\equiv d\pmod m$.
\begin{exes}
\item $a+c\equiv b+d\pmod m$
\item $ac\equiv bd\pmod m$
\item $a^n\equiv b^n\pmod m$ for all $n\geq 1$.    
\end{exes}
\end{ex}


% .....................................
\section{Primes}
\begin{df}
Let $n$ be an integer greater than or equal to~$2$.
Then $n$ is \definend{prime} if its only divisors are itself and~$1$,
else it is \definend{composite}.  
\end{df}

\begin{ex}\pord
An integer $n>1$ is composite if and only if it has factors $a,b$ such that
$1<a,b<n$.  
\end{ex}

\begin{ex}
\begin{exes}
\item Every number greater than $1$ has a prime divisor.
\item Every number $n>1$ has a prime divisor~$p$ with $p\leq \sqrt{n}$.
\item Show that the inequality cannot be made strict.     
\end{exes}
\end{ex}

\begin{ex}[Euclid's Theorem]
There are infinitely many primes.  
\end{ex}

\begin{ex} Suppose that $p$ is a prime.
\begin{exes}
\item If  $p\divides ab$ then either $p\divides a$ or
$p\divides b$.
\item If $p\divides a_1\cdot a_2\cdots a_n$ then 
$p$ divides at least one~$a_i$.    
\end{exes}
\end{ex}

\begin{ex}[Fundamental Theorem of Arithmetic]
\begin{exes}
\item Any number $n>1$ can be written as a product of primes
$n=p_1^{e_1}p_2^{e_2}\cdots p_k^{e_k}$.
\item That factorization is essentially unique:~if 
$n=p_1^{e_1}p_2^{e_2}\cdots p_k^{e_k}$ and we write the primes
in ascending order $p_1<p_2<\cdots<p_k$ then any other
factorization of~$n$ into a product of primes in ascending order will give the
same result.      
\end{exes}
\end{ex}

\begin{ex} The second item is a classic consequence of the Fundamental Theorem.
\begin{exes}
\item Prove that if a number is a square then in its prime factorization 
  each prime occurs an even number of times.
\item Prove that $\sqrt{2}$ is irrational.
\item Generalize the first item to any power.
\item Generalize the second item.    
\end{exes}
\end{ex}


% .....................................
\section{Common divisors and multiples}

\begin{df}
An integer is a \definend{common divisor} of two other integers if it
divides both of them.
The \definend{greatest common divisor} of two integers 
is the largest integer that is their common divisor,
except that we take the greatest common divisor of $0$ and $0$ 
to be $0$.
\end{df}

We write $\gcd(a,b)$ for the greatest common divisor.

\begin{ex}
Generalize the definition to sets of integers.
\end{ex}

\begin{ex}
We will focus on the two integer case.
\begin{exes} 
\item For any two number $a,b$ the greatest common divisor exists.
\item $0\leq \gcd(a,b)\leq\min(\absval{a},\absval{b})$.
\item If either $a$ or $b$ is nonzero then 
  $0< \gcd(a,b)\leq\min(\absval{a},\absval{b})$.
\item If $a>0$ then $\gcd(a,0)=a$.
\item $\gcd(a,a)=\absval{a}$
\item $\gcd(a,b)=\gcd(b,a)$
\item $\gcd(a,b)=\gcd(\absval{a},\absval{b})$
\end{exes}  
\end{ex}

\begin{ex}\pord
\begin{exes}
\item $\gcd(ca,cb)=c\cdot\gcd(a,b)$
\item $\gcd(a+b,c)=\gcd(a,c)+\gcd(b,c)$
\item $\gcd(ab,c)=\gcd(a,c)\cdot\gcd(b,c)$    
\end{exes}
\end{ex}

\begin{ex}
Where the prime factorizations are $a=p_1^{e_1}\cdots p_n^{e_n}$
and $b=p_1^{f_1}\cdots p_n^{f_m}$, in the prime factorization of
$\gcd(a,b)$ the exponent of $p_i$ is $\min(e_i,f_i)$.
\end{ex}

\begin{df}
Two numbers are \definend{relatively prime} if their greatest common
divisor is $1$.  
\end{df}

\begin{ex}
\begin{exes}
\item $\gcd(a,b)=1$ iff $\gcd(b,a)=1$
\item If $\gcd(a,b)=d$ then $\gcd(a/d, b/d)=1$.
\item Is the converse true?    
\end{exes}
\end{ex}

\begin{ex}\notetext{Euclid's algorithm}
Prove that if $a=bq+r$ then $\gcd(a,b)=\gcd(b,r)$.  
\end{ex}

The algorithm associated with this result finds the greatest common
divisor.  For instance, to find $\gcd(803,154)$ compute that
$803=154\cdot 5+33$ and apply the result to get that 
$\gcd(803,154)=\gcd(154,33)$.
Now since $154=33\cdot 4+22$ the result gives
$\gcd(154,33)=\gcd(33,22)$.
By eye we spot the answer of $11$ but continuing as though we hadn't noticed,
we get $33=22\cdot 1+11$ so $\gcd(33,22)=\gcd(22,11)$.
Then $22=11\cdot 2+0$ tells us that $\gcd(22,11)=11$, so the algorithm 
terminates.

\begin{ex}
Use Euclid's Algorithm.
\begin{exes}
\item $\gcd(15,25)$
\item $\gcd(48,732)$  
\end{exes}
\end{ex}

\begin{df}
A number $c$ is a \definend{linear combination} of two others $a$ and~$b$
if it has the form $c=a\cdot m+b\cdot n$ for some $m,n\in\Z$.  
\end{df}

\begin{ex}\notetext{B\'ezout's Lemma}
\begin{exes}
\item Any common divisor of $a$ and~$b$ is a divisor of 
any linear combination of the two.
\item The greatest common divisors of two numbers is a linear combination of 
those two.
\item The greatest common divisor of two numbers is the smallest positive
number that is a linear combination of the two.
\end{exes}
\end{ex}

\begin{ex}
If $a\divides bc$ and $a$ is relatively prime to $b$ then $a\divides c$.  
\end{ex}

\begin{df}
The \definend{least common multiple} of two numbers is the smallest number
that is a multiple of each.
\end{df}

\begin{ex} We write $\lcm(a,b)$.
\begin{exes}
\item The least common multiple of any two integers exists.
\item Generalize to sets of integers.
\item $\lcm(a,b)=\lcm(b,a)$.
\end{exes}
\end{ex}

\begin{ex}
Where the prime factorizations are $a=p_1^{e_1}\cdots p_n^{e_n}$
and $b=p_1^{f_1}\cdots p_n^{f_m}$, 
in the prime factorization of $\lcm(a,b)$ the exponent of 
$p_i$ is $\max(e_i,f_i)$.
\end{ex}

\begin{ex}
Prove that $\lcm(a,b)=ab/\!\gcd(a,b)$.
\end{ex}











%===================================================
\chapter{Sets}
\begin{df}
A \definend{set} is a collection of objects that is definite\Dash every 
object is either
definitely in the set or definitely not in the set.
An object~$x$ that is in a set~$A$ is an \definend{element}
or \definend{member}
of the set, written $x\in A$
(to denote that $x$ is not an element of $A$ write~$x\notin A$).
Two sets are equal if they have the same elements.
\end{df}
\noindent (Read $\in$ as ``is an element of'' rather than ``in'' since the 
latter reading causes confusion with 
the subset relation defined below.)

We often define a set by either listing or describing its elements.
Thus the set containing the primes less than ten could be written
either as $\set{2, 3, 5, 7}$ or as~$\set{p<10\suchthat \text{$p$ is prime}}$.

\begin{ex} \pord
\begin{exes}
\item $\set{1,3,5}=\set{5,3,1}$    
\item $\set{2, 4, 6}=\set{2, 4, 6, 4}$    
\item $\set{1, 3}=\set{n\in\N\suchthat n<5}$ 
\item $0\in\Z+$   
\item $4\in\set{n\in\N\suchthat n^2<50}$
\end{exes}
\end{ex}

\begin{df}
The set with no elements is the \definend{empty set\/} $\emptyset$.  
\end{df}

\begin{df}
The set~$B$ is a \definend{subset} of the set~$A$
if every element of $B$ is an element of~$A$,
that is, if $x\in B$ implies that~$x\in A$.
It is denoted $B\subseteq A$.
\end{df}

\begin{ex} \pord.
\begin{exes}
\item $\set{1,3,5}\subseteq\set{1,3, 5, 7, 9}$
\item $\set{1, 3, 5}\in\set{1, 3, 5, 7, 9}$   
\item $\set{1,3,5}\subseteq\setbuilder{n\in\N}{\text{$n$ is prime}}$
\item $\emptyset\subseteq\set{1, 2, 3, 4}$
\item $\N\subseteq\Z$
\item $\set{2}\in\set{1, \set{2}, 3}$
\item $\set{2}\subseteq\set{1, \set{2}, 3}$
\end{exes}
\end{ex}

\begin{ex} Prove for all sets $A$.
\begin{exes}
\item $A\subseteq A$ and $\emptyset\subseteq A$
\item The empty set is unique: if $A$ is empty and $B$ is empty then $A=B$.
\end{exes}
\end{ex}

\begin{ex}
Prove that if $A\subseteq B$ and~$B\subseteq C$ then~$A\subseteq C$.  
\end{ex}

\begin{ex} For each, give example sets or show it is not possible.
\begin{exes}
\item $A\subseteq B$, $B\not\subseteq C$, $A\subseteq C$
\item $A\not\subseteq B$, $B\not\subseteq C$, $A\subseteq C$
\item $A\not\subseteq B$ $B\subseteq C$, $A\subseteq C$    
\end{exes}
\end{ex}

We will always work inside of a \definend{universal set}.
For instance, while we are doing number theory we may consider 
the set of objects less than $100$,
but in the context of that discussion the universal set is $\Z$ 
so we are considering the set of integers less than $100$.

\begin{df}
The \definend{characteristic function} of a set~$A$ is a map
$\charfcn{A}$ (whose domain is the universal set) such that
$\charfcn{A}(x)=1$ for $x\in A$, and $\charfcn{A}(x)=0$ for $x\notin A$.  
\end{df}






% .....................................
\section{Operations}

\begin{df}
Let $A$ be a set.
The \definend{complement} of $A$, denoted $\setcomp{A}$, is the 
set of objects that are not elements of~$A$.  
\end{df}

One reason for working inside a universal set is that it makes the complement
operation easier. 
For instance, in a discussion of number theory where $A$ is the set of 
things that are less than~$100$ we can take the complement and we needn't 
worry about picking up unwanted objects such as $\pi$.

\begin{ex}
Prove that the complement of the complement is the original set
  $\setcomp{\setcomp{A}}=A$. 
\end{ex}

\begin{df}
Let $A$ and $B$ be sets.
The set of elements 
from either~$A$ or~$B$ is the \definend{union}
$A\union B=\setbuilder{x}{\text{$x\in A$ or $x\in B$}}$.  
The set of elements 
from both~$A$ and~$B$ is the \definend{intersection} 
$A\intersection B=\setbuilder{x}{\text{$x\in A$ and $x\in B$}}$.  
\end{df}

\begin{ex}
Give the formula relating $\charfcn{A}$ and $\charfcn{B}$ to
  $\charfcn{A\intersection B}$.
Do the same for union.
Do the same for complementation.
\end{ex}


Picture set operations with \definend{Venn diagrams}.
\begin{center}
  \grf{asy/venn_union.pdf}{$A\union B$}
  \hspace*{3em}
  \grf{asy/venn_int.pdf}{$A\intersection B$}
  \hspace*{3em}
  \grf{asy/venn_comp.pdf}{$\setcomp{A}$}
\end{center}
In each diagram
the region inside the rectangle depicts the universal set and the 
region inside each circle depicts each of the sets.
On the left the part in mauve shows 
the union as containing all of the two sets joined, 
the middle shows the intersection
containing only the region common to both,
and on the right the mauve region is all but the set.

\begin{df}
The \definend{difference} of two sets is $A-B=\setbuilder{x\in A}{x\notin B}$.  
The \definend{symmetric difference} is 
$A\symdiff B=(A-B)\union(B-A)$.
\end{df}

\begin{ex}
Draw the Venn diagram for difference and symmetric difference.  
\end{ex}

\begin{ex} \pord.
\begin{exes}
\item If $A\subseteq X$ then $X-A$ is the same as $\setcomp{A}$ where
$X$ is the universal set.     
\item $A-B\subseteq A$
\item $A-B=A\intersection\setcomp{B}$
\item For all pairs of sets, $A-B=B-A$.
\item For all pairs of sets, $A\symdiff B=B\symdiff A$.
\end{exes}
\end{ex}

\begin{ex}\notetext{DeMorgan's Laws}
\begin{exes}
\item $A\union\emptyset=A$ and $A\intersection\emptyset=\emptyset$
\item \notetext{Idempotence} $A\union A=A$ and $A\intersection A=A$  
\item $A\intersection B\subseteq A\subseteq A\union B$  
\item \notetext{Commutativity}
   $A\union B=B\union A$ and $A\intersection B=B\intersection A$
\item \notetext{Associativity} 
  $(A\union B)\union C=A\union (B\union C)$
  and $(A\intersection B)\intersection C=A\intersection (B\intersection C)$
\item 
  $\setcomp{A\union B}=\setcomp{A}\intersection\setcomp{B}$
  and 
  $\setcomp{A\intersection B}=\setcomp{A}\union\setcomp{B}$
\item \notetext{Distributivity} 
$A\union (B\intersection C)=(A\union B)\intersection (A\union C)$
 and $A\intersection (B\union C)=(A\intersection B)\union (A\intersection C)$
\end{exes}
\end{ex}

\begin{df}
Two sets are \definend{disjoint} if their intersection is empty.  
\end{df}

\begin{ex}
\pord: $A\intersection B$ and $A-B$ are disjoint.  
\end{ex}

\begin{df}
For a finite set~$A$, the \definend{order} $|A|$ is the number of elements.
\end{df}

% \begin{ex}
% For finite sets, if $A\subseteq B$ then $|A|\leq |B|$
% \end{ex}

\begin{df}
For a set~$A$ the \definend{power set} is $\powerset(A)$ is the set of all
subsets of~$A$.
\end{df}

\begin{ex} List each.
\begin{exes}
\item $\powerset(\set{a,b})$   
\item $\powerset(\set{a,b,c})$   
\item $\powerset(\set{a})$   
\item $\powerset(\emptyset)$   
\end{exes}
\end{ex}

\begin{ex} Prove.
\begin{exes}
% \item $\powerset(A)\subseteq \powerset(B)$ iff $A\subseteq B$  
\item $\powerset(A)=\powerset(B)$ iff $A= B$  
\item Where~$A$ is a finite set, $|\powerset(A)|=2^{|A|}$.    
\end{exes}
\end{ex}

\begin{ex} \pord:
if $A\subseteq B$ then $\powerset(A)\subseteq\powerset(B)$.  
\end{ex}




% .....................................
\section{Cartesian product}

\begin{df}
The \definend{sequence\/} 
formed from $x_0$, $x_1$, \ldots, $x_{n-1}$ is
denoted $\seq{x_0, x_1, \ldots, x_{n-1}}$.
Two sequences $\seq{x_0, x_1, \ldots, x_{n-1}}$ and
$\seq{y_0, y_1, \ldots, y_{n-1}}$ are equal if and only if
they have the same members in the same order:
$x_0=y_0$, $x_1=y_1$, \ldots, $x_{n-1}=y_{n-1}$. 
The \definend{length} $\lh(\seq{x_0, x_1, \ldots, x_{n-1}})$
is the number of entries, $n$.
\end{df}

\begin{ex}\pord.
\begin{exes}
\item $\set{3, 4, 5}=\set{4, 3, 5}$
\item $\sequence{3, 4, 5}=\sequence{4, 3, 5}$
\item $\set{3, 4, 4, 5}=\set{4, 3, 5}$
\item $\sequence{3, 4,4,5}=\sequence{3,4,5}$  
\end{exes}
\end{ex}

\begin{df}
For sets $A_0$, $A_1$, \ldots, $A_{n-1}$ the \definend{Cartesian product}
is the set of all length~$n$ sequences
$A_0\times A_1\times \cdots \times A_{n-1}
  =\setbuilder{\sequence{a_0,a_1,\ldots,a_{n-1}}}{\text{$a_0\in A_0$, and \ldots, and~$a_{n-1}\in A_{n-1}$}}$.

If the sets are the same then 
a sequence of length~$2$ may be called an \definend{ordered pair} and 
written with parentheses $(x_0,x_1)$
(similarly we have ordered triples, $4$-tuples, etc.).
In this case we often write $A\times\cdots\times A$ 
as~$A^n$.
\end{df}

Thus, we may write 
the Cartesian coordinate system for the plane as
$\R^2=\setbuilder{(x,y)}{x,y\in\R}$.

\begin{ex}
\pord:  $\N^2\subseteq \Z^2$.  
\end{ex}

\begin{ex}
\begin{exes}
\item $A\times \emptyset=\emptyset$.
  What about $\emptyset\times A$?
\item $A\times B=\emptyset$ iff $A=\emptyset$ or~$B=\emptyset$
\item Prove that there are sets so that $A\times B\neq B\times A$.
\item Under what circumstances is $A\times B=B\times A$?
\item Is $(A\times B)\times C$ equal to $A\times (B\times C)$?
\item $A\times B\subseteq A^\prime\times B^\prime$ if and only if
  $A\subseteq A^\prime$ and $B\subseteq B^\prime$.
\end{exes}
\end{ex}

\begin{ex} We can ask about the interaction between Cartesian product and 
other set operations.
\begin{exes}
\item $(A\union B)\times C=(A\times C)\union(B\times C)$
\item What happens for intersection?
\item $(A\times B)\union (C\times D)\subseteq ()A\union C\times (B\union D)$
\item What happens for intersection?
\item What happens for complement?
\end{exes}
\end{ex}








%===================================================
\chapter{Functions, relations}
\begin{df}
A \definend{function}~$f$ (or \definend{map} or \definend{morphism}) 
from \definend{domain} set~$D$
to \definend{codomain} set~$C$, written $\map{f}{D}{C}$,
consists of the two sets along with a \definend{graph}, 
a set of pairs $(d,c)\in D\times C$ that is 
\definend{well-defined}: for each $d\in D$ there is
exactly one $c\in C$ such that $(d,c)\in f$. 

Thus a function associates each element~$d$ from the domain,
called an \definend{argument} (or~\definend{input}),
with an element~$c$ from the codomain, 
called a~\definend{value} (or~\definend{output}). 
We may also say the $c$ is the \definend{image} of $d$, 
or $d$ \definend{maps to}~$c$, and write $f(d)=c$.
\end{df}

\noindent Functions are equal only if they have the same domain, codomain,
and graph.

Draw a function by showing the domain and codomain as blobs 
or \definend{beans}, with either a simple arrow from the domain to
the codomain,  
or with an barred arrow from elements of the domain to elements of the
codomain.
\begin{center}
  \grf{asy/bean_fcn.pdf}{$\map{f}{D}{C}$}
  \hspace{10em}
  \grf{asy/bean_fcn_elets.pdf}{$a\mapsunder{}f(a)$ and $b\mapsunder{}f(b)$}
\end{center}
In place of $\map{f}{D}{C}$ you may see $D\mapsvia{f} C$ and
in place of $f(d)=c$ you may see $d\mapsunder{f}c$.

\begin{ex} \pord{} that each is a function.\label{FindFunctions}
\begin{exes}
\item $D=\set{0,1,2}$, $C=\set{3,4,5}$,
  $G=\set{(0,3), (1,4), (2,5)}$    
\item $D=\set{0,1,2}$, $C=\set{3,4,5}$,
  $G=\set{(0,3), (1,4), (2,3)}$    
\item $D=\set{0,1,2}$, $C=\set{3,4,5}$,
  $G=\set{(0,3), (1,4)}$    
\item $D=\set{0,1,2}$, $C=\N$,
  $G=\set{(0,3), (1,3), (2,3)}$    
\item $D=\N$, $C=\N$,
  $G=\set{(0,3), (1,4), (2,5)}$    
\item $D=\set{0,1,2}$, $C=\set{3,4,5}$,
  $G=\set{(0,3), (1,4), (0,5)}$    
\item $D=\N$, $C=\N$,
  $G=\setbuilder{(d,c)\in D\times C}{c=d^2}$    
\end{exes}
\end{ex}

We often ignore the distinction between a function and its graph, and 
think of the function as the set of ordered pairs 
(the graph is only in the definition because the codomain cannot be recovered
from the set of ordered pairs).

\begin{ex}
Let $D$ and $C$ be finite sets.
How many maps are there from $D$ to $C$?
\end{ex}

\begin{df}
A \definend{constant function} $\map{f}{D}{C}$ has  
$f(d_0)=f(d_1)$ for all $d_0,d_1\in D$.
% \end{df}\begin{df}
The \definend{identity function} $\map{\identity}{D}{D}$ has
$\identity(d)=d$ for all $d\in D$.
\end{df}

A function may have multiple arguments; one example is 
$(x,y)\mapsunder{f}x^2-2y^2$.
We would write this map $\map{f}{\R^2}{\R}$ as 
$f(x,y)$ rather than
$f((x,y))$.
The number of arguments is the function's \definend{arity}.

\begin{df}
The \definend{range} of $\map{f}{D}{C}$
is $\setbuilder{y\in C}{\text{there is a $x\in D$ such that $f(x)=y$}}$.
\end{df}

\begin{ex}
For each function of \cref{FindFunctions} find the range.  
\end{ex}

\begin{df}
Let $\map{f}{D}{C}$.
The \definend{restriction} of $f$ to $B\subseteq D$ is
$\map{\restrictionmap{f}{B}}{B}{C}$ whose action is given by 
$\restrictionmap{f}{B}(b)=f(b)$ for all $b\in B$.
Where $g$ is a restriction of~$f$, we say that
$f$ is an \definend{extension} of~$g$.

The \definend{image} $f(b)$ of $B$ under $f$ is the range of 
the function $\restrictionmap{f}{B}$.
\end{df}

\begin{df}
Let $\map{f}{D}{C}$.
The \definend{inverse image of the element}~$c\in C$ is
the set $f^{-1}(c)=\setbuilder{d\in D}{f(d)=c}$.
The \definend{inverse image of the set}~$A\subseteq C$
is the set $f^{-1}(A)=\setbuilder{d\in D}{f(d)\in A}$   
\end{df}

\begin{ex}
Prove that $f^{-1}(A)$ is the union of the sets $f^{-1}(c)$ for all $c\in A$.
\end{ex}





% .....................................
\section{Composition}

\begin{df}
The \definend{composition} of
the two functions
$\map{f}{D}{C}$ and $\map{g}{C}{B}$ 
is $\map{\composed{g}{f}}{D}{B}$ given by 
$\composed{g}{f}(d)=g(f(d))$.
\end{df}

Observe that although when reading from left to right $g$ comes first, 
it is the function that is applied second. 
Observe also that the codomain of~$f$ is the domain of~$g$.

\begin{ex} Let $D=\set{0,1,2}$, $C=\set{a,b,c,d}$, 
and $B=\set{\alpha,\beta,\gamma}$
Suppose that $\map{f}{D}{C}$ is given by $0\mapsunder{} a$, $1\mapsunder{} c$, 
$2\mapsunder{} d$ and that $\map{g}{C}{B}$
is given by 
$a\mapsunder{}\alpha$, $b\mapsunder{} \beta$, and $c\mapsunder{}\gamma$.
\begin{exes}
\item Compute $\composed{g}{f}$ on all arguments or show the composition
  function is not defined.
\item Compute $\composed{f}{g}$ on all arguments or show it is not defined.
\item Find the range of $f$, $g$, and $\composed{g}{f}$.    
\end{exes}
\end{ex}

\begin{ex}
Let $\map{f}{\R}{\R}$ be $f(x)=x^2$ and let $\map{g}{\R}{\R}$ be~$g(x)=3x+1$.
Find the domain, codomain, and a formula for each of
$\composed{g}{f}$ and $\composed{f}{g}$.  
\end{ex}

\begin{ex}\notetext{Properties of composition} Prove each.
\begin{exes}
\item Composition need not be commutative.
\item\notetext{Associativity} 
  $\composed{h}{(\composed{g}{f})}=\composed{(\composed{h}{g})}{f}$    
% \item $g(f(A))=\composed{g}{f}(A)$
% \item $(\composed{g}{f})^{-1}(C)=f^{-1}(g^{-1}(C))$
\end{exes}
\end{ex}





% .....................................
\section{Inverse}

The definition of function specifies that for every argument there is 
exactly one associated value.

\begin{df}
A function is \definend{one-to-one} (or an \definend{injection}) 
if for each value there is at most
one associated argument, that is, if $f(d_0)=f(d_1)$ implies that $d_0=d_1$
for elements $d_0,d_1$ of the domain.
A function is \definend{onto} (or a \definend{surjection}) 
if for each value there is at least
one associated argument, that is, if for each element $c$ of the codomain
there exists an element $d$ of the domain such that $f(d)=c$.
A function that is both one-to-one and onto is a 
\definend{correspondence} (or \definend{bijection}, or \definend{permutation}).
\end{df}

\begin{ex}
If $D$ and~$C$ are finite sets and there is a 
correspondence~$\map{f}{D}{C}$
then the two sets
have the same number of elements.  
\end{ex}

\begin{ex} Prove.
\begin{exes}
\item A composition of one-to-one functions is one-to-one.
\item A composition of onto functions is onto.
\item A composition of correspondences is a correspondence.    
\end{exes}
\end{ex}

\begin{ex} 
\begin{exes}
\item If $g\circ f$ is onto then $g$ is onto.
\item If $g\circ f$ is one-to-one then $f$ is one-to-one.
\item Do the other two cases hold?     
\end{exes}
\end{ex}

\begin{df}
A function \definend{inverse} to $\map{f}{D}{C}$ is 
$\map{f^{-1}}{C}{D}$ such that 
$\composed{f^{-1}}{f}$ is the identity map on~$D$ and
$\composed{f}{f^{-1}}$ is the identity map on~$C$.
\end{df}

\begin{ex} 
\begin{exes}
\item Let $\map{s}{\R^+}{\R}$ be~$s(x)=x^2$ and let
  $\map{r}{\R^+}{\R^+}$ be $r(x)=\sqrt{x}$.
  Show that $r$ is inverse to $s$.    
\item Let $\map{f}{\Z}{\Z}$ be~$f(x)=x+3$ and let
  $\map{g}{\Z}{\Z}$ be $g(x)=x-3$.
  Show that $g$ is inverse to $f$.
\item Show that $s$ is inverse to $r$, and that $f$ is inverse to~$g$.
\item Let $\map{h}{\N}{\N}$ be the function that returns
  $n+1$ if $n$ is even, and returns $n-1$ if $n$ is odd.
  Find a function inverse to~$h$.
\item Show that $\map{t}{\R}{\R}$ given by $t(x)=x^2$
  has no inverse.
\end{exes}
\end{ex}

\begin{ex}
\begin{exes}
\item Find a pair of maps $\map{f}{D}{C}$ and $\map{g}{C}{D}$
  such that $\composed{g}{f}$ is the identity but $\composed{f}{g}$
  is not.
  (We say that $g$ is a \definend{right inverse} of~$f$, or what is the
  same thing, that $f$ is a \definend{left inverse} of~$g$).
\item Find a pair of maps $\map{f}{D}{C}$ and $\map{g}{C}{D}$
  such that $\composed{f}{g}$ is the identity but $\composed{g}{f}$
  is not.
\end{exes}
\end{ex}

\begin{ex} Prove each.
\begin{exes}
\item If a function has an inverse then that inverse
  is unique.
\item A function has an inverse if and only if that 
  function is a correspondence.
\item The inverse of a correspondence is a correspondence.  
\item $(g\circ f)^{-1}=f^{-1}\circ g^{-1}$
\end{exes}  
\end{ex}





% .....................................
\section{Relations}
\begin{df}
A \definend{relation} on sets $A_0$, \ldots, $a_{n-1}$ is a subset
$R\subseteq A_0\times \cdots \times A_{n-1}$. 
If all of the sets are the same $a_0=A$, \ldots, $A_{n-1}=A$
then we call it a relation on~$A$.
The number of sets~$n$ is the \definend{arity} of the relation.
and say it is \definend{$n$-ary}.
If the arity is~$2$ then it is a \definend{binary relation}.
In this case, if $R=\setbuilder{(x,y)}{\text{$x\in X$ and $y\in Y$}}$
then where $(x,y)\in R$ we say $x$ is \definend{$R$-related} to~$y$
and sometimes write $xRy$.
\end{df}

\begin{ex}
\begin{exes}
\item List five elements of the relation
  $R=\setbuilder{(x,y)\in \R^2}{x+2=y}$
\item Consider less-than $<$ as a binary relation on $\N$.
  List five elements.
\item List five elements of the relation
  $\setbuilder{(x,y,z)\in\N^3}{x^2+y^2=z^2}$
\item Show that if $\map{f}{D}{C}$ is a function then 
  its graph is a relation.
\item Let $\map{f}{D}{C}$ be a function.
  Show that 
  $R_f=\setbuilder{(x,y)}{f(x)=f(y)}$
  is a binary relation.
  Where $f(x)=x^2$ with domain and codomain~$\R$,
  list five elements. 
\end{exes}
\end{ex}

\begin{df} 
Let $R$ be a binary relation on a set~$X$.
The relation is \definend{reflexive} if $(x,x)\in R$ for all $x\in X$.
The relation is \definend{symmetric} if $(x,y)\in R$ implies that
$(y,x)\in R$ for all $x,y\in R$.
The relation is \definend{transitive} if 
$(x,y)\in R$ and $(y,z)\in R$ implies that 
$(x,z)\in R$ for all elements $x,y,z\in R$.
A relation that satisfies all three conditions is an
\definend{equivalence relation}.  
\end{df}

\begin{ex}   \pord{} that each is reflexive, symmetric, and transitive.
\begin{exes}
\item The ``goes into'' relation
  $D=\setbuilder{(d,m)\in\Z^2}{d\divides m}$.
\item
  For any set~$A$ the \definend{diagonal relation} on $A$ 
  is $\setbuilder{(x,x)}{x\in A}$. 
\item
  Where $A$ is a set,
  $E=\setbuilder{(x,y)\in A\times\powerset(A)}{x\in y}$.
\end{exes}
\end{ex}

\begin{ex} For each, find a binary relation satisfying the condition.
\begin{exes}
\item Not reflexive.
\item Not transitive.
\item Reflexive and symmetric but not transitive.
\item Transitive but not symmetric or reflexive.
\item Reflexive and transitive but not symmetric.
\item Symmetric and transitive but not reflexive. 
\end{exes}
\end{ex}

\begin{ex}
Two integers have the same \definend{parity} if they are both odd, or 
both even.
Show that the binary relation 
$P=\setbuilder{(x,y)\in\Z^2}{\text{$x$ and $y$ have the same parity}}$  
is an equivalence.
\end{ex}

\begin{ex}
Fix a divisor $q\in\N-\set{0}$.
Show that the relation 
$\mathord\sim=\setbuilder{(m,n)\in\N^2}{m\equiv n\pmod q}$  
is an equivalence.
\end{ex}

\begin{df}
If $R$ is an equivalence relation on~$X$ then instead of $(x,y)\in R$
we may write $x\equiv y\pmod R$.  
\end{df}

\begin{ex} \label{RationalsAsEqClasses}
\begin{exes}
\item Let $X=\Z\times (\N-\set{0})$.
Show that 
$Q=\setbuilder{\big((p,q),(n,d)\big)\in X\times X}{pm=qn}$
is an equivalence.
\item List five elements of $Q$.
\end{exes}
\end{ex}

\begin{ex} \label{PlaneLinesAsClasses}
\begin{exes}
\item Let $X$ be the set of lines in the Euclidean plane and consider
the relation
$R=\setbuilder{(\ell_0,\ell_1)\in X^2}{\text{the two are parallel}}$. 
Show that is an equivalence.
\item List five elements of $R$.
\item Fix a vertical line $\ell$ (for instance, the $x$-axis 
  $\setbuilder{(x,0)}{x\in\R}$).
  List five elements $X$ related to~$\ell$.
\end{exes}
\end{ex}

\begin{df}
Let $R$ be an equivalence relation on~$X$.
The \definend{equivalence class} of $x\in X$ is the set
$\eqclass{x}=\setbuilder{y\in X}{y\equiv x\pmod R}$.   
\end{df}

\begin{ex} Describe the equivalence classes.
\begin{exes}
\item \cref{RationalsAsEqClasses}
\item \cref{PlaneLinesAsClasses}
\end{exes}
\end{ex}

\begin{df}
A \definend{partition} of a set $X$ is a 
set of nonempty subsets of~$X$ such that every element $x\in X$ 
is in exactly one of these subsets.
That is, a set~$P$ is a partition of $X$ if and only if 
none of its elements is the empty set,
it \definend{covers}~$X$
(the union of the elements of $P$ is equal to~$X$),
and the elements of $P$ are \definend{pairwise disjoint}
(that is, $p_i\intersection p_j=\emptyset$ for $i\neq j$).
\end{df}

\begin{ex} Prove that the following are equivalent statements for $x,y\in X$.
\begin{exes}
\item $x\equiv y\pmod R$
\item $\eqclass{x}=\eqclass{y}$    
\item $\eqclass{x}\intersection\eqclass{y}\neq \emptyset$
\end{exes}
\end{ex}

Let $X$ be a set. Consider an equivalence relation~$R$ on that set, 
and the partition $P=\setbuilder{\eqclass{x}}{x\in X}$.
We say that the partition is  
the \definend{induced} by the relation or that
the relation \definend{arises} from the partition. 

\begin{ex}
Suppose that $\map{f}{D}{C}$ is onto.
Show that the relation
$R=\setbuilder{(d_0,d_1)\in D^2}{f(d_0)=f(d_1)}$ 
is an equivalence on~$D$. 
Show that the set of inverse images 
$P=\setbuilder{f^{-1}(c)}{c\in C}$ is a partition of the domain.
\end{ex}

\begin{df}
A binary relation~$R$ is \definend{antisymmetric} if
$(x,y)\in R$ and $(y,x)\in R$ implies that $x=y$.
A relation is a \definend{partial ordering} if it is 
reflexive, antisymmetric, and transitive.  
\end{df}

\begin{ex}
Can a relation be both symmetric and antisymmetric?  
\end{ex}

\begin{ex} Prove each.
\begin{exes}
\item The usual less than or equal to relation~$\leq$ on 
the real numbers is a partial order.
\item The relation `divides' on $\N$ is a partial order.
\item For any set~$A$ the relation $\subseteq$ on $\powerset(A)$ is
a partial order.
\end{exes}
\end{ex}





%===================================================
\chapter{Infinity}

\begin{df}
Two sets have the \definend{same cardinality} 
(or are \definend{equinumerous}) if there is a 
correspondence from one to the other.
We write $A\sim B$.   
\end{df}

\begin{ex} Prove.
\begin{exes}
\item The relation $\sim$ is an equivalence.
\item For all natural numbers $n$, any pair of sets with~$n$ elements
have the same cardinality .
\item Fix a set~$A$ with $n$~elements.
  For any $m\neq n$, a set with $m$ elements does not have the same
  cardinality as~$A$.
\end{exes}
\end{ex}

\begin{df}
A set is \definend{finite} if it has~$n$ elements for some $n\in\N$.
Otherwise the set is \definend{infinite}.   
\end{df}

\begin{df}
A set is \definend{denumerable} if it has the same cardinality 
as $\N$.
A set is \definend{countable} if it is either finite or denumerable.
\end{df}

\begin{ex}  Prove.
\begin{exes}
\item The set~$\N\times\N$ is countable.
\item The set of integers is countable.
\end{exes}
\end{ex}

\begin{ex} Prove that the following are equivalent for a set~$A$.
\begin{exes}
\item $A$ is countable
\item $A$ is empty or there is an onto function from $\N$ to~$A$
\item There is a one-to-one function from $A$ to~$\N$    
\end{exes}
\end{ex}

\begin{ex} Prove.
\begin{exes}
\item The set of rational numbers is countable.
\item The union of countably many sets is countable.
\item The set of finite sequences of elements from a countable set is countable.
\end{exes}
\end{ex}

\begin{ex}
\begin{exes}
\item The set $\powerset{\N}$ is not countable.
\item The set of real numbers is not countable.  
\end{exes}
\end{ex}




%===================================================
\appendix
\chapter{Peano axioms}
\newcommand{\axiom}[1]{\textsc{Axiom~#1}\hspace{.5ex}}

Particularly in the 
first chapter a person struggles with when to consider a 
statement sufficiently justified 
and soon comes to wonder what the axioms are like.
Here we give the most often used axiom system for the natural numbers, to
convey a sense of that. 

This system was 
introduced by Dedekind in 1888 and tuned by Peano in 1889.
In addition to the usual set and logical symbols such as $=$ and $\in$,
with the traditional properties, 
our language will use at least two symbols, $0$ and $S$, whose 
properties are limited by the conditions below.

\begin{ax}[Existence of a natural number]
The constant~$0$ is a natural number.
\end{ax}

\begin{ax}[Arithmetical properties]
The \definend{successor} function~$S$ has these properties.
\begin{exes}
\item \notetext{Closure} For all $a\in\N$, its successor~$S(a)$ is also 
  a natural number.
\item \notetext{One-to-one} For all $a,b\in\N$, if $S(a)=S(b)$ then $a=b$.
\item \notetext{Almost onto}
  For all $a\in\N$, if $a\neq 0$ then there is a $b\in\N$ with $S(b)=a$.
  In contrast, no $c\in\N$ has $0$ as a successor.  
\end{exes}
\end{ax}


These properties give infinitely many natural numbers:
$0$, $S(0)$, $S(S(0))$, etc.
Of course, the notation $0$, $1$, $2$, etc., is less clunky.

\begin{ax}[Induction]
  Suppose that $K$ is a set satisfying that both (i)~$0\in K$
  and (ii)~for all $n\in\N$, if $n\in K$ then $S(n)\in K$.
  Then $K=\N$.
\end{ax}

We prefer an induction variant that changes
condition~(ii) to: for all $n\in\N$, 
if $0\in K$ and $1\in K$ and \ldots\ and $n\in K$ then $S(n)\in K$.
These two variants are inter-derivable but while ours is 
more awkward to state, it is sometimes more convenient to use.

From those axioms we can for instance 
define addition by recursion using successor
\begin{equation*}
  \add(a,n)=
  \begin{cases}
    a             &\text{if $n=0$}  \\
    S(\add(a,m))  &\text{if $n=S(m)$} 
  \end{cases}
\end{equation*}
and then define multiplication by recursion using addition.
\begin{equation*}
  \mul(a,n)=
  \begin{cases}
    0             &\text{if $n=0$}  \\
    \add(\mul(a,m),a)  &\text{if $n=S(m)$} 
  \end{cases}
\end{equation*}
\end{document}
