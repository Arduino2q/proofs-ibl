\chapter{Infinity}

Recall Exercise~\ref{CorrespondingSetsHaveSameNumberOfElements},
that if two sets correspond then they have the same number of elements.

\begin{df}
Two sets have the \definend{same cardinality} 
(or are \definend{equinumerous}) if there is a 
correspondence from one to the other.
We write $A\sim B$.   
\end{df}

\begin{df}
A set is \definend{finite} if it has~$n$ elements for some $n\in\N$,
that is, if it has the same cardinality as 
$\setbuilder{i\in\N}{i< n}=\set{0,1,\ldots,n-1}$.
Otherwise the set is \definend{infinite}.   
% \end{df}\begin{df}
A set is \definend{denumerable} if it has the same cardinality 
as $\N$.
A set is \definend{countable} if it is either finite or denumerable.
\end{df}

\begin{problem} \label{ex:EquinumeruousIsEquivalence}
Prove that the relation~$\sim$ is an equivalence.
\begin{answer}
  To show that the relation~$\sim$ is reflexive 
  suppose that $A$ is a set and note that the 
  identity function~$\map{\identity}{A}{A}$ is a correspondence.
  To show that the relation is symmetric recall 
  Exercise~\ref{PropertiesOfInverses}, that
  any correspondence~$\map{f}{A}{B}$ has an inverse~$\map{f^{-1}}{B}{A}$
  that is also a correspondence.

  To prove that the relation~$\sim$ is transitive assume that 
  $A\sim B$ and~$B\sim C$.
  Then there is a correspondence~$\map{f}{A}{B}$ and a 
  correspondence $\map{g}{B}{C}$.
  The composition $\map{\composed{g}{f}}{A}{C}$ is a correspondence
  by Exercise~\ref{InteractionOneToONeOntoWithComposition}.
  Thus $A\sim B$.
\end{answer}
\end{problem}

\begin{problem}  \label{ex:IntegersCountable}
Prove.
\begin{exes}
\begin{exercise} 
  The set of integers is countable.
\end{exercise}
\begin{answer}
  We will define a map $\map{f}{\N}{\Z}$ and just note that it is clearly a 
  correspondence.
  The map alternate between positive and negative values:
  $0\mapsunder{}0$, $1\mapsunder{}1$, $2\mapsunder{}-1$, 
  $3\mapsunder{}2$, $4\mapsunder{}-2$, $5\mapsunder{}3$, \ldots\,.
  More formally, $f(n)=-n/2$ if $n$ is even and $f(n)=(n+1)/2$ if~$n$ is odd.   
\end{answer}
\begin{exercise} 
  The set~$\N\times\N$ is countable.
\end{exercise}
\begin{answer}
  As in the prior item we will define a map
  $\map{f}{\N}{\N\times\N}$ and only note that it is a correspondence, 
  without going through the verification details.
  
  The function alternates between columns in the same way that
  Jackie Chan, when up against multiple bad guys, alternates among his 
  attackers.
  Thus $0\mapsunder{} (0,0)$, $1\mapsunder{}(0,1)$, $2\mapsunder{}(1,0)$, 
  $3\mapsunder{}(0,2)$, 
  $4\mapsunder{}(1,1)$, $5\mapsunder{}(2,0)$, $6\mapsunder{}(0,3)$, \ldots. 

  \remark a sketch helps here.
  The values associated with successive arguments trace out the 
  diagonals of the array~$\N\times\N$.
  \begin{center}
  \newcommand{\chanpoint}[1]{\makebox[0em]{$(#1)$}}
  \newcommand{\chanarrow}{{\color{darkii} $\searrow$}}
  \setlength{\unitlength}{0.5in}
  \begin{picture}(4,4)
    \put(0,0){\chanpoint{0,0}}
    \put(1,0){\chanpoint{1,0}}
    \put(2,0){\chanpoint{2,0}}
    \put(3,0){\chanpoint{3,0}}
    \put(3.5,0){\ldots}
    \put(0,1){\chanpoint{0,1}}
    \put(1,1){\chanpoint{1,1}}
    \put(2,1){\chanpoint{2,1}}
    \put(2.75,1){\ldots}
    \put(0,2){\chanpoint{0,2}}
    \put(1,2){\chanpoint{1,2}}
    \put(2,2){\chanpoint{2,2}}
    \put(2.75,2){\ldots}
    \put(0,3){\chanpoint{0,3}}
    \put(1,3){\chanpoint{1,3}}
    % \put(2,3){\ldots}
    \put(0,3.6){$\vdots$}
    \put(1,3.6){$\vdots$}
    \put(2.5,3.1){$\iddots$}
    % arrows
    \put(0.5,0.5){\chanarrow}
    \put(0.5,1.5){\chanarrow}
    \put(1.5,0.5){\chanarrow}
    \put(0.5,2.5){\chanarrow}
  \end{picture}
  \end{center}  
\end{answer}
\end{exes}
% \begin{exes}
% \item The set of integers is countable.
% \item The set~$\N\times\N$ is countable.
% \end{exes}
% \begin{ans}
% \begin{exes}
% \item We will define a map $\map{f}{\N}{\Z}$ and just note that it is clearly a 
%   correspondence.
%   The map alternate between positive and negative values:
%   $0\mapsunder{}0$, $1\mapsunder{}1$, $2\mapsunder{}-1$, 
%   $3\mapsunder{}2$, $4\mapsunder{}-2$, $5\mapsunder{}3$, \ldots\,.
%   More formally, $f(n)=-n/2$ if $n$ is even and $f(n)=(n+1)/2$ if~$n$ is odd. 
% \item  As in the prior item we will define a map
%   $\map{f}{\N}{\N\times\N}$ and only note that it is a correspondence, 
%   without going through the verification details.
  
%   The function alternates between columns in the same way that
%   Jackie Chan, when up against multiple bad guys, alternates among his 
%   attackers.
%   Thus $0\mapsunder{} (0,0)$, $1\mapsunder{}(0,1)$, $2\mapsunder{}(1,0)$, 
%   $3\mapsunder{}(0,2)$, 
%   $4\mapsunder{}(1,1)$, $5\mapsunder{}(2,0)$, $6\mapsunder{}(0,3)$, \ldots. 

%   \remark a sketch helps here.
%   The values associated with successive arguments trace out the 
%   diagonals of the array~$\N\times\N$.
%   \begin{center}
%   \newcommand{\chanpoint}[1]{\makebox[0em]{$(#1)$}}
%   \newcommand{\chanarrow}{{\color{darkii} $\searrow$}}
%   \setlength{\unitlength}{0.5in}
%   \begin{picture}(4,4)
%     \put(0,0){\chanpoint{0,0}}
%     \put(1,0){\chanpoint{1,0}}
%     \put(2,0){\chanpoint{2,0}}
%     \put(3,0){\chanpoint{3,0}}
%     \put(3.5,0){\ldots}
%     \put(0,1){\chanpoint{0,1}}
%     \put(1,1){\chanpoint{1,1}}
%     \put(2,1){\chanpoint{2,1}}
%     \put(2.75,1){\ldots}
%     \put(0,2){\chanpoint{0,2}}
%     \put(1,2){\chanpoint{1,2}}
%     \put(2,2){\chanpoint{2,2}}
%     \put(2.75,2){\ldots}
%     \put(0,3){\chanpoint{0,3}}
%     \put(1,3){\chanpoint{1,3}}
%     % \put(2,3){\ldots}
%     \put(0,3.6){$\vdots$}
%     \put(1,3.6){$\vdots$}
%     \put(2.5,3.1){$\iddots$}
%     % arrows
%     \put(0.5,0.5){\chanarrow}
%     \put(0.5,1.5){\chanarrow}
%     \put(1.5,0.5){\chanarrow}
%     \put(0.5,2.5){\chanarrow}
%   \end{picture}
%   \end{center}
% \end{exes}
% \end{ans}
\end{problem}

\begin{problem} \label{ex:OntoFromNatsToSetEquivToCountable}
Prove that the following are equivalent for a set~$A$:
(i)~$A$ is countable
(ii)~$A$ is empty or there is an onto function from $\N$ to~$A$
(iii)~there is a one-to-one function from $A$ to~$\N$.    
\begin{answer}
We will show that (i) implies~(ii), which implies~(iii), and which
implies~(i).

First the (i) implies~(ii) proof.
If~$A$ is countable then there are two cases: either~$A$ is finite 
or~$A$ is denumerable.
If~$A$ is finite then it is either empty or nonempty.
The empty case is trivial so suppose that $A$ is finite and nonempty
and let $a$ be an element of~$A$.
Then there is a correspondence~$\map{f}{\set{0,1,\ldots,|A|-1}}{A}$,
so define $\map{\hat{f}}{\N}{A}$ extending~$f$ by
$\hat{f}(i)=f(i)$ if $i\in\set{0,1,\ldots,|A|-1}$, and 
$\hat{f}(i)=a$ otherwise.
The extension~$\hat{f}$ is onto because $f$ is onto.
The other case, that~$A$ is denumerable, by definition has a 
correspondence between~$\N$ and~$A$.

Next is the (ii) implies~(iii) proof.
If~$A$ is empty then the empty function is a map with domain~$A$ and
codomain~$\N$ that is vacuously one-to-one.
The other possibility is that there is an onto function~$\map{f}{\N}{A}$.
To get a map~$g$ with domain~$A$ and codomain~$\N$ that is one-to-one
define~$g(a)$ to be the least number $n\in\N$ such that~$f(n)=a$; 
at least one such number exists because~$f$ is onto.
Clearly~$g$ is one-to-one.

Finally for the (iii) implies~(i) proof.
Let~$\map{f}{A}{\N}$ be one-to-one.
Let $B=\range(A)\subseteq\N$.
Then the function~$\map{\hat{f}}{A}{B}$ given by~$\hat{f}(a)=f(a)$ is a 
correspondence (it only eliminates unused elements of the codomain).
Thus $A\sim B$.
Because Exercise~\ref{ex:EquinumeruousIsEquivalence} shows that the 
relation~$\sim$ is an equivalence, we will be done if we show that $B$~is
countable.

So suppose that~$B$ is not finite.
Recall that~$B\subseteq\N$ so we can define a map
$\map{g}{\N}{B}$ by:
take $g(j)$ to be the $j$-th largest element of~$B$.
This map is onto because of the assumption that~$B$ is not finite, and it 
is one-to-one by construction (if $g(j_0)=g(j_1)$ then the $j_0$-th largest
element of~$B$ is the same as the $j_1$-th largest element, which is only
possible if $j_0=j_1$).
\end{answer}
\end{problem}

\begin{problem} 
  Prove that the set of rational numbers is countable.
\begin{answer}
  We will show that the set on nonnegative rationals
  $Q=\setbuilder{q\in\Q}{q\geq 0}$ is countable.
  With that set countable, a function that alternates sign like the one in 
  Exercise~\ref{ex:IntegersCountable} will show that~$\Q$ is countable.

  Consider the function $\map{g}{\N\times\N}{\Q}$ where $g(p,q)$
  is the rational number $(p+1)/(q+1)$.
  It is not quite onto; every rational number is the image of some
  $(p,q)$ except for $0$.

  The Cartesian product $\N\times\N$ is denumerable so there is a 
  correspondence $\map{f}{\N}{\N\times\N}$.
  Define a function $\map{h}{\N}{\Q}$ by $h(i)=0$ if $i=0$ and $h(i)=f(i-1)$
  if $i>0$.
  The range is $\range(h)=\range(g)\union\set{0}$ so~$h$ is onto.
  By Exercise~\ref{ex:OntoFromNatsToSetEquivToCountable} the 
  rationals~$\Q$ form a countable set.
\end{answer}
\end{problem}

\begin{problem}  Prove that each of these infinite sets is not countable.
\begin{exes}
\begin{exercise} 
  $\powerset(\N)$ 
\end{exercise}
\begin{answer}
  We will show that no $\map{f}{\N}{\powerset(\N)}$ is onto.
  Consider $R=\setbuilder{n\in\N}{n\notin f(n)}$.
  Clearly $R\subseteq \N$ and so $R\in\powerset(\N)$.
  We will show that there is no $i\in\N$ such that $R=f(i)$.

  For contradiction assume that $R=f(i)$.
  Consider whether $i\in R$.
  If $R=f(i)$ and $i\in R$ then that contradicts the definition of $R$ as the 
  collection of integers~$n$ such that $n\notin f(n)$.
  The other $R=f(i)$ possibility is that $i\notin R$, which 
  is also a contradiction
  because then $i$~satisfies the defining criteria for membership in~$R$, namely
  that $i\notin R$, and therefore $i\in R$.
  Either $R=f(i)$ case gives a contradiction and so $R\neq f(i)$.
\end{answer}
\begin{exercise} 
  $\R$
\end{exercise}
\begin{answer}
  We will show that there is a one-to-one map~$\map{f}{\powerset(\N)}{\R}$.
  If $\R$ were countable\Dash if there were a one-to-one 
  function $\map{g}{\R}{\N}$\Dash then 
  that would give a one-to-one function~$\composed{g}{f}$ 
  from $\powerset(\N)$ to~$\N$,
  contradicting the prior item.

  We define the action of~$f$ on a set~$A\in\powerset(\N)$ by
  giving an associated real number between $0$ and~$1$.
  We will give this number's binary expansion, instead of its more usual
  decimal expansion. 
  Let this number be such that its $i$-th binary place is $\charfcn{A}(i)$,
  that is, the $i$-th binary place is $1$ if $i\in A$ and is $0$ otherwise. 
  This function is one-to-one because two unequal sets differ in an 
  element~$a\in\N$, and the two image numbers then differ in the $a$-th place.  
\end{answer}
\end{exes}
% \begin{exes}
% \item $\powerset(\N)$ 
% \item $\R$
% \end{exes}
% \begin{ans}
% \begin{exes}
% \item We will show that no $\map{f}{\N}{\powerset(\N)}$ is onto.
%   Consider $R=\setbuilder{n\in\N}{n\notin f(n)}$.
%   Clearly $R\subseteq \N$ and so $R\in\powerset(\N)$.
%   We will show that there is no $i\in\N$ such that $R=f(i)$.

%   For contradiction assume that $R=f(i)$.
%   Consider whether $i\in R$.
%   If $R=f(i)$ and $i\in R$ then that contradicts the definition of $R$ as the 
%   collection of integers~$n$ such that $n\notin f(n)$.
%   The other $R=f(i)$ possibility is that $i\notin R$, which 
%   is also a contradiction
%   because then $i$~satisfies the defining criteria for membership in~$R$, namely
%   that $i\notin R$, and therefore $i\in R$.
% Either $R=f(i)$ case gives a contradiction and so $R\neq f(i)$.
% \item We will show that there is a one-to-one map~$\map{f}{\powerset(\N)}{\R}$.
%   If $\R$ were countable\Dash if there were a one-to-one 
%   function $\map{g}{\R}{\N}$\Dash then 
%   that would give a one-to-one function~$\composed{g}{f}$ 
%   from $\powerset(\N)$ to~$\N$,
%   contradicting the prior item.

%   We define the action of~$f$ on a set~$A\in\powerset(\N)$ by
%   giving an associated real number between $0$ and~$1$.
%   We will give this number's binary expansion, instead of its more usual
%   decimal expansion. 
%   Let this number be such that its $i$-th binary place is $\charfcn{A}(i)$,
%   that is, the $i$-th binary place is $1$ if $i\in A$ and is $0$ otherwise. 
%   This function is one-to-one because two unequal sets differ in an 
%   element~$a\in\N$, and the two image numbers then differ in the $a$-th place.
% \end{exes}
% \end{ans}
\end{problem}
