\chapter*{Preface}

This is a course in mathematical proof. 
It is for math majors who are US sophomores, although since
it requires only high school mathematics
it can be used with first year students.



\medskip
\noindent\textsc{Approach.}
This course is inquiry-based (sometimes called Moore method 
or discovery method).
That is, this text is a sequence of exercises,
along with definitions and a few 
remarks.
The students together work through the mathematics by
proving statements or sometimes by providing examples or counterexamples.
They grapple directly with the mathematics\Dash the instructor only 
lightly guides, while students pledge not to use outside sources\Dash
talking out misunderstandings, 
sometimes stumbling in the dark, and sometimes
having beautiful flashes of insight.
For these students, with this material,
this is the best way to develop mathematical maturity.
Besides, it is a lot of fun.


\medskip
\noindent\textsc{Topics.}
We start with elementary number theory, not logic and sets, 
for the same reason
that the baseball team's annual practice starts with tossing the ball and 
not with reading the rulebook.
Math majors take readily to proving things about
divisibility and primes, enjoying the 
intellectual challenge and verbal exchange, 
whereas a first month of background material can be less of an inducement.

But the background in logic and sets is good stuff also and 
students are on board once they see where it is going.
In the second and third chapters we do
sets, functions, and relations, now with the
intellectual habits that we've established at the start.

% There are some topics, such as the definition of implication, 
% that students need even in the
% first chapter.
% My class meetings begin with some people going to the 
% board to write their answers to the homework and 
% I often use these ten minutes for a beamer slide or two 
% discussing issues from
% the current work.
% This just-in-time approach keeps the class centered 
% on proving things while ensuring that they get what
% they need when they need it.
% (The only exception is that we pause briefly to do induction,
% since it is typically new for these students and since it needs 
% their full attention.)



\medskip
\noindent\textsc{Exercises types.}
Some exercises have multiple items; these come in two types.
If the items are labeled \textsc{A}, \textsc{B}, etc., 
then I expect that in class each item answer will be presented 
by a different student.
If the labels are (i), (ii), etc., then I expect that one  
student will answer all of the items.
(Often these exercises are examples.)



\medskip
\noindent\textsc{License.}
This material is Free; see \url{http://joshua.smcvt.edu/proofs}.
This includes the \LaTeX{} source so that instructors can tailor the
material to their students.  
% Also on that page are the slides that I use
% and, for instructors new to this approach, 
% a brief description of the classroom approach that works for me.


\vspace{\fill}
\noindent\parbox{.95\textwidth}{\raggedright\textit{At the first meeting of the class Moore would define the basic terms and either challenge the class to discover the relations among them, or, depending on the subject, the level, and the students, explicitly state a theorem, or two, or three. Class dismissed. Next meeting: "Mr Smith, please prove Theorem 1. Oh, you can't? Very well, Mr Jones, you? No? Mr Robinson? No? Well, let's skip Theorem 1 and come back to it later. How about Theorem 2, Mr Smith?" Someone almost always could do something. If not, class dismissed. It didn't take the class long to discover that Moore really meant it, and presently the students would be proving theorems and watching the proofs of others with the eyes of eagles.}\hspace{1.5em}--Paul Halmos}

\vspace{.2in}
\noindent\parbox{.95\textwidth}{\textit{It's a kind of art that may change lives.}\hspace{1.5em}--Peter Schjeldahl}
 
\vspace*{\fill}
\begin{flushright}
  \begin{tabular}{@{}l@{}}
  Jim Hef{}feron  \\
  Saint Michael's College  \\
  Colchester, Vermont USA \\
  2013-Spring
  \end{tabular}
\end{flushright}
