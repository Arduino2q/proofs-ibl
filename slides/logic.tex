% see: https://groups.google.com/forum/?fromgroups#!topic/comp.text.tex/s6z9Ult_zds
\makeatletter\let\ifGm@compatii\relax\makeatother 
\documentclass[10pt,t]{beamer}
\usefonttheme{professionalfonts}
\usefonttheme{serif}
\PassOptionsToPackage{pdfpagemode=FullScreen}{hyperref}
\PassOptionsToPackage{usenames,dvipsnames}{xcolor}
% \DeclareGraphicsRule{*}{mps}{*}{}
% \usepackage{../ibl}
\usepackage{../ibl}
\usepackage{present}
% \usepackage{xr}\externaldocument{../ibl} % read refs from .aux file
% \usepackage{catchfilebetweentags}
\usepackage{etoolbox} % from http://tex.stackexchange.com/questions/40699/input-only-part-of-a-file-using-catchfilebetweentags-package
\makeatletter
\patchcmd{\CatchFBT@Fin@l}{\endlinechar\m@ne}{}
  {}{\typeout{Unsuccessful patch!}}
\makeatother

\mode<presentation>
{
  \usetheme{boxes}
  \setbeamercovered{invisible}
  \setbeamertemplate{navigation symbols}{} 
}
\addheadbox{filler}{\ }  % create extra space at top of slide 
\hypersetup{colorlinks=true,linkcolor=blue} 

\title[Proof foundations] % (optional, use only with long paper titles)
{Proof foundations}

\author{{\small Jim Hef{}feron}}
\institute{
  \texttt{http://joshua.smcvt.edu/proofs}
}
\date{}


\subject{Proof foundations}
% This is only inserted into the PDF information catalog. Can be left
% out. 

\begin{document}
\begin{frame}
  \titlepage
\end{frame}



% ============================================================
\section{The need for proof}
\begin{frame}{In Mathematics we prove things}

That the base angles of an isoceles triangle are equal is obvious
to a person with a mathematical aptitude.
Another statement that is obvious to such a person
is that natural numbers factor into a product of primes. 

But is the Pythagorean Theorem, that in a right triangle the 
square of the length of the hypoteneuse is equal to the sum
of the squares of the other two sides, also obvious?
Does it not need to be firmly established?

One of the characteristics of our subject is 
that we prove new results from already established 
statements.  
\end{frame}


\begin{frame}{Why we must prove things}
\begin{itemize}
\item
At first glance the polynomial $n^2+n+41$ seems to output only primes.
Here are the outputs that are less than $100$.
\begin{center} \small
  \begin{tabular}{r|rrrrrrrr}
    $n$        &$0$  &$1$  &$2$  &$3$  &$4$  &$5$  &$6$  &$7$ \\ \cline{2-9}
    $n^2+n+41$ &$41$ &$43$ &$47$ &$53$ &$61$ &$71$ &$83$ &$97$
  \end{tabular}
\end{center}
But that pattern breaks down;  
for $n=41$ the polynomial $41^2+41+41$ is clearly 
divisible by~$41$.

\pause
\item 
We can count a positive integer's prime factors. 
For instance $18=2\cdot 3\cdot 3$ has three.
If it has an even number of them it is of \textit{even} type, otherwise it
is of \textit{odd} type (the number $1$ is of odd type).
In 1919 P\`olya conjectured that below any~$n$ the even types never
outnumber the odd types. 
\begin{center} \small
  \makebox[0.9\textwidth]{
  \begin{tabular}{r|rrrrrrrrrr}
    $n$  &$1$  &$2$  &$3$  &$4$  &$5$  &$6$  &$7$ &$8$ &$9$ \\ \cline{2-10}
    type &odd  &odd  &odd  &even &odd  &even &odd &odd &even
  \end{tabular}}
\end{center}
In 1962 that Lehman found the 
counterexample $906\,180\,359$.
\end{itemize}
\end{frame}
% >>> def prime_factors(n):
% ...     i = 2
% ...     factors = []
% ...     while i * i <= n:
% ...         if n % i:
% ...             i += 1
% ...         else:
% ...             n //= i
% ...             factors.append(i)
% ...     if n > 1:
% ...         factors.append(n)
% ...     return factors
% >>> for i in range(41):
% ...     print i, i*i+i+41, prime_factors(i*i+i+41)
% ... 
% 0 41 [41]
% 1 43 [43]
% 2 47 [47]
% 3 53 [53]
% 4 61 [61]
% 5 71 [71]
% 6 83 [83]
% 7 97 [97]
% 8 113 [113]
% 9 131 [131]
% 10 151 [151]
% 11 173 [173]
% 12 197 [197]
% 13 223 [223]
% 14 251 [251]
% 15 281 [281]
% 16 313 [313]
% 17 347 [347]
% 18 383 [383]
% 19 421 [421]
% 20 461 [461]
% 21 503 [503]
% 22 547 [547]
% 23 593 [593]
% 24 641 [641]
% 25 691 [691]
% 26 743 [743]
% 27 797 [797]
% 28 853 [853]
% 29 911 [911]
% 30 971 [971]
% 31 1033 [1033]
% 32 1097 [1097]
% 33 1163 [1163]
% 34 1231 [1231]
% 35 1301 [1301]
% 36 1373 [1373]
% 37 1447 [1447]
% 38 1523 [1523]
% 39 1601 [1601]
% 40 1681 [41, 41]


% ============================================================
\section{Elements of logic}

%..........
\begin{frame}
  \frametitle{Propositions}

A \alert{proposition} is an assertion that has a truth value, 
either `true' or `false'.

\pause
These are propositions: ``$2+2=4$'' 
and ``Two circles in the plane intersect in either zero points, one point,
two points, or all of their points.''

\pause
These are not propositions: ``$3+5$''
and ``$x$ is not prime.''
\end{frame}



%..........
\begin{frame}[<+->]
  \frametitle{Negation}

Prefixing a proposition with \alert{not} inverts its truth value.

``It is not the case that $3+3=5$''
is true.

``It is not the case that $3+3=6$''
is false.

\pause
\bigskip
So the truth
of \textit{not}~$P$ depends only on the truth value of~$P$.
Specifically, the truth value of \textit{not}~$P$ is 
the opposite of the truth value of~$P$.  
We say it is a \alert{unary logical operator} or 
a \alert{unary boolean function} since it takes one input, a truth value, 
and yields as output a truth value.


\end{frame}


%..........
\begin{frame}
  \frametitle{Conjunction, disjunction}

A proposition consisting of the word \alert{and}
between two sub-propositions is true if
the two halves are true.

`$3+1=4$ and 
$3-1=2$'
is true

`$3+1=4$ and 
$3-1=1$'
is false

`$3+1=5$ and 
$3-1=2$'
is false

\pause
\bigskip
A compound proposition constructed with \alert{or}
between two sub-propositions is true if at least one half
is true.

`$2\cdot 2=4$ or 
$2\cdot 2\neq 4$'
is true

`$2\cdot 2=3$ or 
$2\cdot 2\neq 4$'
is false

`$2\cdot 2=4$ or 
$3+1=4$'
is true

\bigskip
\pause
So
`and' and `or'
are \alert{binary logical operators}.
\end{frame}





%..........
\begin{frame}
  \frametitle{Truth Tables}

Describe the action of operators with \alert{truth tables}. 
\begin{center}
  \begin{tabular}[t]{c|c}
    $P$  &$\neg P$  \\ \hline
    $F$  &$T$  \\
    $T$  &$F$
  \end{tabular}
  \qquad
  \begin{tabular}[t]{cc|cc}
    $P$  &$Q$  &$P\wedge Q$  &$P\vee Q$  \\ \hline
    $F$  &$F$  &$F$          &$F$  \\
    $F$  &$T$  &$F$          &$T$  \\
    $T$  &$F$  &$F$          &$T$  \\
    $T$  &$T$  &$T$          &$T$  
  \end{tabular}
\end{center}
Write $\neg P$ for `not~$P$', 
$P\wedge Q$ for `$P$~and~$Q$',
and $P\vee Q$ for `$P$~or~$Q$'.  

\pause
One advantage of this
notation is that it allows formulas of a complexity that would be awkward in
an English sentence. 
For instance,
$(P\vee Q)\wedge \neg(P\wedge Q)$ is hard to express in 
English.
\end{frame}




%..........
\begin{frame}
Sometimes we prefer using $0$ for~$F$ and $1$ for~$T$.
One reason for the preference is that on 
the left side of the tables the rows make the ascending binary numbers.
\begin{center}
  \begin{tabular}[t]{c|c}
    $P$  &$\bar{P}$  \\ \hline
    $0$  &$1$  \\
    $1$  &$0$
  \end{tabular}
  \qquad
  \begin{tabular}[t]{cc|cc}
    $P$  &$Q$  &$P\cdot Q$  &$P+ Q$  \\ \hline
    $0$  &$0$  &$0$          &$0$  \\
    $0$  &$1$  &$0$          &$1$  \\
    $1$  &$0$  &$0$          &$1$  \\
    $1$  &$1$  &$1$          &$1$  
  \end{tabular}
\end{center}
\pause
In this context \textit{not}~$P$ is symbolized $\bar{P}$.
Note that $\bar{P}=1-P$.

The table makes clear why `$P$~and~$Q$' is symbolized with a 
multiplication dot~$P\cdot Q$.

For `$P$~or~$Q$' the plus sign is the best symbol because 
\textit{or} is accumulation of the truth value~$T$.
\end{frame}







%..........
\begin{frame}
  \frametitle{Other operators: Exclusive Or}

Disjunction models
sentences meaning `and/or'.
In contrast,
``Eat your dinner or no dessert'',
``Live free or die'', and
``Let me go or the hostage gets it'' all mean `one or the other, but not both'.
\begin{center}
  \begin{tabular}{cc|c}
    $P$  &$Q$  &$P$ \text{\small\textsc{XOR}} $Q$  \\ \hline
    $F$  &$F$  &$F$          \\
    $F$  &$T$  &$T$          \\
    $T$  &$F$  &$T$          \\
    $T$  &$T$  &$F$     
  \end{tabular}
\end{center}
\end{frame}




%..........
\begin{frame}
  \frametitle{Other operators: Implies}

We model ``if $P$ then $Q$'' this way.
\begin{center}
  \begin{tabular}{cc|c}
    $P$  &$Q$  &$P \rightarrow Q$  \\ \hline
    $F$  &$F$  &$T$          \\
    $F$  &$T$  &$T$          \\
    $T$  &$F$  &$F$          \\
    $T$  &$T$  &$T$     
  \end{tabular}
\end{center}
Here $P$ is the \emph{antecedent} while $Q$ is the 
\emph{consequent}. 
\end{frame}




%..........
\begin{frame}
  \frametitle{Other operators: Bi-implication}

Model ``$P$ if and only if $Q$'' with this.
\begin{center}
  \begin{tabular}{cc|c}
    $P$  &$Q$  &$P \leftrightarrow Q$  \\ \hline
    $F$  &$F$  &$T$          \\
    $F$  &$T$  &$F$          \\
    $T$  &$F$  &$F$          \\
    $T$  &$T$  &$T$     
  \end{tabular}
\end{center}
Mathematicians sometimes write ``iff''.
\end{frame}



%..........
\begin{frame}
  \frametitle{All binary operators}

We can lists all of the binary logical operators.
\begin{center} \small
    \begin{tabular}{cc|c}
       $P$  &$Q$  &$P$ $\alpha_0$ $Q$  \\ \hline
       $F$  &$F$  &$F$          \\
       $F$  &$T$  &$F$          \\
       $T$  &$F$  &$F$          \\
       $T$  &$T$  &$F$     
     \end{tabular}
     \quad\begin{tabular}{cc|c}
       $P$  &$Q$  &$P$ $\alpha_1$ $Q$  \\ \hline
       $F$  &$F$  &$F$          \\
       $F$  &$T$  &$F$          \\
       $T$  &$F$  &$F$          \\
       $T$  &$T$  &$T$     
     \end{tabular}                
    \quad\ldots\quad                      
    \begin{tabular}{cc|c}
      $P$  &$Q$  &$P$ $\alpha_{15}$ $Q$  \\ \hline
      $F$  &$F$  &$T$          \\
      $F$  &$T$  &$T$          \\
      $T$  &$F$  &$T$          \\
      $T$  &$T$  &$T$ 
    \end{tabular}
\end{center}
\pause
\bigskip
These are the unary ones.
\begin{center} \small
  \begin{tabular}{cccc}
    \begin{tabular}{c|c}
       $P$  &$\beta_0 P$  \\ \hline
       $F$  &$F$          \\
       $T$  &$F$     
     \end{tabular}
    &\begin{tabular}{c|c}
       $P$  &$\beta_1 P$  \\ \hline
       $F$  &$F$          \\
       $T$  &$T$     
     \end{tabular}               
    &\begin{tabular}{c|c}
       $P$  &$\beta_2 P$  \\ \hline
       $F$  &$T$          \\
       $T$  &$F$     
     \end{tabular}
    &\begin{tabular}{c|c}
       $P$  &$\beta_3 P$  \\ \hline
       $F$  &$T$          \\
       $T$  &$T$     
     \end{tabular} 
  \end{tabular}
\end{center}

\pause
\bigskip
A zero-ary operator is constant so there are two:
$T$ and $F$.
\end{frame}




%..........
\begin{frame}
  \frametitle{Evaluating complex statements}
  We can calculate how the output results 
  depend on the values of the input variables.
  \pause
  Here is the input-output relationship for
  $(P\rightarrow Q)\wedge (P\rightarrow R)$.
  \begin{center}
    \begin{tabular}{ccc|ccc}
      $P$  &$Q$  &$R$ &$P\rightarrow Q$ &$P\rightarrow R$ &$(P\rightarrow Q)\wedge (P\rightarrow R)$  \\ \hline
      $F$  &$F$  &$F$  &$T$  &$T$  &$T$   \\
      $F$  &$F$  &$T$  &$T$  &$T$  &$T$   \\
      $F$  &$T$  &$F$  &$T$  &$T$  &$T$   \\
      $F$  &$T$  &$T$  &$T$  &$T$  &$T$   \\
      $T$  &$F$  &$F$  &$F$  &$F$  &$F$   \\
      $T$  &$F$  &$T$  &$F$  &$T$  &$F$   \\
      $T$  &$T$  &$F$  &$T$  &$F$  &$F$   \\
      $T$  &$T$  &$T$  &$T$  &$T$  &$T$      
    \end{tabular}
  \end{center}
\end{frame}




%..........
\begin{frame}
  \frametitle{Tautology, Satisfiability, Equivalence}
  A formula is a \emph{tautology} if it evaluates to $T$ for every value
  of the variables.
  A formula is \emph{satisfiable} if it evaluates to $T$ for at least one
  value of the variables.

  \pause
  Two propositional expressions are \alert{logically equivalent} if they
  give the same input-output relationship. 
  Check that the expressions 
  $E_0$ and~$E_1$ are equivalent by using truth tables to
  verify that  
  $E_0\leftrightarrow E_1$ is a tautology.  

  For instance, $P\wedge Q$ and $Q\wedge P$ are equivalent.
  Another example is that $P\rightarrow Q$ and $\neg Q\rightarrow \neg P$ 
  are equivalent.
\end{frame}






% %..........
% \begin{frame}
%   \frametitle{Topic: Sheffer stroke}
% This single operator
% \begin{center}
%   \begin{tabular}{cc|c}
%     $P$  &$Q$  &$P \mathbin{\text{{\small\textsc{nand}}}} Q$  \\ \hline
%     $F$  &$F$  &$T$          \\
%     $F$  &$T$  &$T$          \\
%     $T$  &$F$  &$T$          \\
%     $T$  &$T$  &$F$     
%   \end{tabular}
% \end{center}
% can be used to produce any truth table.
% For instance, $P\mathbin{\text{\small\textsc{nand}}}P$ is equivalent to $\neg P$.
% \pause
% The \textsc{nor} operation is also complete.
% \end{frame}




% %..........
% \begin{frame}
%   \frametitle{Topic: McCarthy Or}
% The \texttt{or} operator used in computer languages is somewhat
% like the one discussed here but it is \emph{short-circuited}.
% In this expression
% \begin{center}
%   \texttt{if isEven(x) or isPrime(x)}
% \end{center}
% if \texttt{x} is even then \texttt{isPrime(x)} will not be evaluated.
% \end{frame}




%..........
\begin{frame}
  \frametitle{Non-obvious lines in the implication table}
\begin{center}
  \begin{tabular}{cc|c}
    $P$  &$Q$  &$P \rightarrow Q$  \\ \hline
    $F$  &$F$  &$T$          \\
    $F$  &$T$  &$T$          \\
    $T$  &$F$  &$F$          \\
    $T$  &$T$  &$T$     
  \end{tabular}
\end{center}
Consider
`if Babe Ruth was president then $1+2=4$'. 
Our definition of implies takes it to be a true statement, 
because its antecedent is false.
\pause 
Now consider 
`if Mallory reached the summit of Everest then life exists on earth.' 
We take this to be true because its consequent is true.
\pause
Why define implication this way?

\pause
Standard mathematical practice defines implication so that
\begin{center}
  if $n$ is a perfect square then $n$ is not prime
\end{center}
is true for all $n\in\N$.
\pause  Take $n=6$ to get that $F\rightarrow T$ must evaluate to $T$.  
\pause Take $n=3$ to get that $F\rightarrow F$ should yield $T$.
\pause For $T\rightarrow T$ take $n=4$.
\end{frame}
% http://www.earlham.edu/~peters/courses/log/mat-imp.htm
\begin{frame}\vspace*{-1ex}
\frametitle{Points about implication}
\begin{center}
  \begin{tabular}{cc|c}
    $P$  &$Q$  &$P \rightarrow Q$  \\ \hline
    $F$  &$F$  &$T$          \\
    $F$  &$T$  &$T$          \\
    $T$  &$F$  &$F$          \\
    $T$  &$T$  &$T$     
  \end{tabular}
\end{center}
\begin{itemize}
\item As noted on the prior slide, 
  the antecedent~$P$ need not be materially connected to the 
  consequent~$Q$.
\pause
\item Also noted there are:
  (1) if the antecedent~$P$ is false then the statement as a whole is true; 
  said to be \alert{vacuously true}
  and (2) if the consequent~$Q$ is true then the statement as a whole is true.
% \pause 
% \item If the antecedent~$P$ is true then the statement as a whole has the
%   same truth value as the consequent.
\pause
\item Truth tables show that $P\rightarrow Q$
  is equivalent to $\neg(P\wedge \neg Q)$, 
  to $\neg P\vee Q$,
  and also equivalent to the \alert{contrapositive}~$\neg Q\rightarrow \neg P$.
\pause
\item 
  On a table in front of you are four cards, 
  marked `A', `B', `0', and~`1'.
  You must verify the truth of the implication, 
  `if a card has a vowel on the one side 
  then it has an even number on the other.'  
  How to do it, turning over the fewest cards?
% \item In ``Fred is Mike's brother's son and therefore Fred is Mike's nephew'' 
%   the statement ``Fred is Mike's nephew'' is a \textit{material consequence} 
%   of ``Fred is Mike's brother's son'' but not a formal consequence. 
%   This is because the validity of the argument depends on the 
%   internal content of 
%   the premise and conclusion, including the meanings of `brother', `son', 
%   and `nephew', rather than on the logical form of the argument.
\end{itemize}
\end{frame}
% See http://www.cs.cornell.edu/Info/People/gries/symposium/clarke.htm


\begin{frame}
\frametitle{Predicates, Quantifiers}
Consider this statement. 
\begin{equation*}
  \text{`if $n$ is odd then $n$ is a perfect square'}
  \tag{$*$}
\end{equation*}
\pause
It involves two clauses, `$n$ is odd' and `$n$ is square',
and for both the truth value depend on the variable.
A \alert{predicate} is a truth-valued function.
An example is the function~$\text{Odd}$ that takes an integer as input and 
yields either $T$ or~$F$, as in~$\text{Odd}(5)=T$.
Another example is $\text{Square}$, as in $\text{Square}(5)=F$.

\pause
A mathematician asserting~($*$) would mean that
it holds for all~$n$.
We denote `for all' by $\forall$ so the statement is formally written
$\forall n\in\N \big[\text{Odd}(n)\rightarrow\text{Square}(n)\big]$.
(It is, of course a false statement.)

\pause
A
\alert{quantifier} describes for how many values of the
variable the clause must be true, in order for the statement as a whole to
be true.
Besudes `for all' 
the other common quantifier is 
`there exists', denoted $\exists$.
The statement
$\exists n\in\N \big[\text{Odd}(n)\rightarrow\text{Square}(n)\big]$
is true.
\end{frame}
\begin{frame}
Examples of statements written with explicit quantifiers.

\begin{itemize}
\item Every number is divisible by $1$.
  \begin{equation*}
    \forall n\in\N\,\big[1\divides n\big]
  \end{equation*}

\pause
\item There are five different powers~$n$ where the equation $2^n-7=a^2$ has a solution.
  \begin{multline*}
    \exists n_0, \ldots, n_4\in\N\, \big[(n_0\neq n_1) 
                                     \wedge (n_0\neq n_2) 
                                     \wedge \cdots 
                                     \wedge (n_3\neq n_4)  \\
                                     \wedge \exists a_0\in\N(2^{n_0}-7=a_0^2)
                                     \wedge\cdots\wedge
                                     \exists a_4\in\N(2^{n_4}-7=a_4^2)\big]
  \end{multline*}

\pause
\item Any two integers have a common multiple.
  \begin{equation*}
    \forall n_0,n_1\in\N\;\exists m\in\N\,
        \big[(n_0\divides m)\wedge (n_1\divides m)\big]
  \end{equation*}

\pause
\item The function~$\map{f}{\R}{\R}$ is continuous at $a\in\R$.
  \begin{equation*}
    \forall \varepsilon >0\;\exists \delta > 0\;\forall x\in\R\,
        \big[(\absval{x-a}<\delta)\rightarrow (\absval{f(x)-f(a)}<\varepsilon)\big]
  \end{equation*}
\end{itemize}
\end{frame}
\begin{frame}
Observe that the negation of a `$\forall$' statement is a `$\exists\,\neg$' 
statement.
For instance, the negation of `every odd number is a perfect square'
\begin{equation*}
  \forall n\in\N\,\big[ \text{Odd}(n)\rightarrow\text{Square}(n)\big]
\end{equation*}
is 
\begin{equation*}
  \exists n\in\N\,\neg\big[ \text{Odd}(n)\rightarrow\text{Square}(n)\big]
\end{equation*}
which is equivalent to this.
\begin{equation*}
  \exists n\in\N\,\big[\text{Odd}(n)\wedge\neg\text{Square}(n)\big]
\end{equation*}
Thus a person could show `every odd number is a perfect square' is false
by finding a number that is both odd and not a square.

\pause
Simililarly the negation of a `$\exists$' statement is a~`$\forall\,\neg$'
statement. 
\end{frame}






%..........
% \begin{frame}
% \ex
% \end{frame}


%...........................
% \begin{frame}
% \ExecuteMetaData[../gr3.tex]{GaussJordanReduction}
% \df[def:RedEchForm]
% 
% \end{frame}
\end{document}
